% Enable warnings about problematic code
\RequirePackage[l2tabu, orthodox]{nag}

\documentclass{WeSTassignment}

% The lecture title, e.g. "Web Information Retrieval".
\lecture{Lecture Title}
% The names of the lecturer and the instructor(s)
\author{%
  Dr.~John~Doe\\{\normalsize\mailto{johndoe@uni-koblenz.de}} \and
  William~Smith\\{\normalsize\mailto{williamsmith@uni-koblenz.de}}
}
% Assignment number.
\assignmentnumber{1}
% Institute of lecture.
\institute{%
  Institute of Web Science and Technologies\\%
  Department of Computer Science\\%
  University of Koblenz-Landau%
}
% Date until students should submit their solutions.
\datesubmission{March 13, 2007, 10:00 a.m.}
% Date on which the assignments will be discussed in the tutorial.
\datetutorial{March 15, 2007}

% Should solutions be printed?
%\solutiontrue

% Set langauge of text.
\setdefaultlanguage[
  variant = american, % Use American instead of Britsh English.
]{english}

% Specify bib file location.
\addbibresource{example_bibliography.bib}

% ==============================================================================
% Document

\begin{document}

\maketitle

You can have text on the title page if you want.

% ------------------------------------------------------------------------------

\section{Assignment Title (10 Points)}

Here goes the text that describes what tasks the students have to perform for
this assignment.

\begin{itemize}
  \item Giving bullet points is preferred because they allow for easier
    overview and can be though of as \emph{ticked-off}.
  \item A footnote is coming up\footnote{Text inside the footnote.}.
  \item Trying out references \footcite{DBLP:conf/leet/CheckowaySR10}.
    Use \texttt{\textbackslash{}footcite} so that the citation appears in the
    footnote, because there is no References section.
    This works because we have set \texttt{style~=~verbose} in the
    \texttt{biblatex} (\url{https://www.ctan.org/pkg/biblatex}) options.
    Another citation coming up \footcite{DBLP:books/sp/Gratzer16}.
\end{itemize}

\inlinesolution{
  Here goes a inline reference solution to the task.
  Solutions are automatically colored red.
}

\begin{solution}
  Use the \texttt{solution}-solution environment\dots

  \dots{}if you want solutions with multiple paragraphs.
\end{solution}

% If you need more elaborate solution logic you can use:
%\ifsolution
%  Only executed when showing solutions.
%  No automatic coloring takes place when using this variant.
%\else
%  Only executed when not showing solutions.
%\fi


% ------------------------------------------------------------------------------


\section{Collision detection (5 Points)}

\emph{This following example exercise is by Stefan Scheglman from the
  \enquote{Introduction to Web Science (WS1516)} course, Assignment 1.}

Write a simple python program that does a simplified simulation of Ethernet with
collision-detection.

\begin{itemize}
  \item Your should simulate networks with 2 up to 100 hosts.
  \item In the simulation, each host should send 8 KBytes of data (just write
    them to the network).
  \item In one simulation step a host can send 1 Bit of data.
  \item Assume that the distance between hosts is always one. (A host occupies
    the network after he finished to send the first bit.)
  \item The minimum frame size is 512 Byte. Assume that only minimal frames are
    send.
  \item Consider the overhead of the Ethernet frame header and footer when
    calculating the total amount of frames to be send for the 8 KBytes of data.
    Also consider that packages and not frames are send. (Advise: check
    \url{https://en.wikipedia.org/wiki/Ethernet_frame})
  \item If a collision is detected (two or more hosts sending at the same time)
    the sending process is reset and the colliding hosts waits for a random time
    slot before they start to send again (according to the collision detection
    algorithm presented in the Web Science MOOC).
  \item At the end of the simulation generate a plot with the number of
    computers on the x-axis and the time (simulation steps to send 8 Kbyte from
    every host) on the y-axis.
\end{itemize}

We provide you with a unfinished \texttt{.py} script in your SVN source folder.
Please use this file to complete your program and fill in the given variables
according to the specifications and values given above.

\makefooter

\end{document}
